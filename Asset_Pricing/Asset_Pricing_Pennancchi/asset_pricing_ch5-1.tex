\documentclass[uplatex,a4paper]{jsarticle}


\usepackage[uplatex]{otf}
\usepackage[dvipdfmx]{graphicx}
\usepackage{tabularx}
\usepackage{float}
\usepackage{geometry}
\usepackage{listings}
\usepackage{amsmath}
\usepackage{amssymb}
\usepackage{mathrsfs}
\usepackage{url}
\usepackage{color}

\title{消費と投資選好に関する多期間離散モデル}
\author{羅洪帥(57165029)}
\date{}

\begin{document}

\maketitle
\hrule
\medskip  %

今までは、単期間もしくは静的なモデルで分析したんですが、本章から期待効用最大化する投資家は
多期間のもしくは動的な消費投資選好を分析する。
多期間の消費投資・ポートフォリオの選好問題は投資家の生涯での貯蓄と投資の配分を最適化する理論を提供するので、
とても意味深いである。
投資家のファイナンシャル・プランニングを指導することを期待される。
また、一期間の平均分散ポートフォリオモデルはCAPMモデルでの資産需要の理論を支持したように、
多期間のポートフォリオ選択は異時点間の資産プライシングの一般的な均衡モデル
に資産需要の理論を提示する。
個人投資家は消費と証券に対する選好モデルと
会社の生産技術モデルを纏めて考えると、
資産価格を決定する経済均衡モデルを導く。

先行研究にはランダム動的計画法を用いた。それは唯一の方法ではないですが、便利で且つ直感的な解決法である。

本章では、投資家は投資期間内に消費投資計画を複数回に決定するモデルを検討する。
投資期間は、投資家の生存期間と解釈し、 いくつの決定期間に分け、投資家は決定期間ごとに消費投資に関して意思を決定する。
このモデルは以前の単期間モデルと比べれば、より多くな内容は含まれている。
単期間モデルでは、決定期間と投資期間は被っている。
多期間問題を一連の単期間問題とみなし、単期間モデルで得た分析結果は流用できる。

本章では、投資家は離散の時間での意思決定すると想定する。本書以降の内容では連続時間での検討をする。
離散時間での検討は簡単で、明確な結論を導出できる。
連続時間なら、連続時間ランダムプロセスで資産のプライシングをすることになる。

これから幾つかの仮定をするが、一番強い仮定は投資家は時間分離型効用関数を持つこと。
問題を解くには、 ランダム動的計画を紹介する。
この解決法では多期間の意思決定問題を一連の単期間の意思決定問題にブレークダウンして、
問題を簡潔化する。
すると、 最後の期間の最適化から、徐々に前倒しで、 現在時点の最適決定を解決する。
最後に対数型効用関数での例を検討する。


\section*{5.1 仮定の整理と記号の準備}
投資家は 消費、n個のリスク資産、無リスク資産から 割合を決定する。
投資家は各資産の価格変化はランダムプロセスであることを把握している。
つまり、完全の競争市場である。
投資家は市場価格を受取しかない、 市場価格を左右することはできない。
一般的には上場資産は流動性を持ち、 投資家にとってはこの仮定は実世界と一致する。
また、取引コストと税金などの"摩擦"がないような市場を想定する。

0からTまでのT個の期間に、投資家は各期間の期初で消費・投資の決定をする。

\subsection*{5.1.1 選好}

投資家は消費水準と最終遺産に決めるの期待効用関数を最大化すると仮定する。
消費水準は$C_t, t = 0,...,T-1$で、最終遺産は$W_T$、第$t$期の期末の富を$W_t$で表記する。
期待効用関数は$E_0[\Upsilon(C_0,C_1,...,C_{T-1},W_T)]$と書ける。効用関数である$\Upsilon$は凹関数
(説明変数が増加すれば、関数は増えますが、増やす分は段々減る)。さらに$\Upsilon$は時間分離型もしくは
時間加法型と仮定する。(つまり、異時点の効用関数はお互いに影響しません)。
$$
E_0[\Upsilon(C_0,C_1,...,C_{T-1},W_T)] =
E_0[\sum_{t=0}^{T-1} U(C_t,t)+B(W_T,T)]
$$
UとBはそれぞれ消費と富水準の増加型の凹関数である。
上式では、第$t$期の効用は当期の消費だげに依存するように仮定。
今後はこの仮定を緩めにする。


\subsection*{5.1.2 富の動向}

投資家は富$W_t$と賃金$y_t$を持ち、期初で消費と貯蓄を配分する。貯蓄とは、リスク資産と無リスク資産間で
ポートフォリオを組むこと。リスク資産$i$の収益を$R_{it}$で表記、無リスク収益を$R_{ft}$、
第$t$期の貯蓄の中で資産$i$の割合は$w_{it}$と表記すれば、
$$
W_{t+1} = (W_t + y_t - C-t) (R_{ft} + \sum_{i=1}^{n} w_{it} (R_{it} - R_{ft})) = S_t R_t
$$
$S_t$は$t$期の貯蓄水準、$W_t+y_t-C_t$。
$R_t$は収益で、$R_{ft} + \sum_{i=1}^{n} w_{it} (R_{it} - R_{ft})$。

ここでは、資産の収益の確率分布については制限してない。資産の収益の分布は時間と共に変化し、異なる時間帯では
異なる確率分布を持つこと。
しかも、無リスク収益も時間帯により、変化する。
つまり、時間の推移により、投資家が直面する投資状況は変わって、 多期間モデルでは、投資家の意思決定は当期の資産収益分布に
影響されるし、将来の資産収益の変化にも影響される。

図5.1が示したように、投資家は毎期が既知の情報がある。第$t$期には、初期の富、今期の賃金収入、来期までの無リスク収益、
それらの情報を$I_t$で表記。$I_t$には当期以前の情報も入ってます。
投資家は$I_t$を持って、来期の無リスク収益と賃金収入の分布を予測する。
これらの元で、投資家は消費水準と各リスク資産の割合を決める。
\subsection*{5.2 多期間モデルの解}
間接(?)効用関数$J(W_t,t)$を定義。

maxは、消費水準と各リスク資産の割合の調整で括弧内の期待値を最大化という意味です。
それは、現在までの情報で、現在の手元の資産を配分すること。

これらの情報(現在までの情報で、現在の手元の資産という意味?)はリスク資産の収益分布の変化もしくは
無リスク収益の変化を反映する。
(全体的に意味不明)
このような変化は外生的だと仮定する。
(投資家の投資行動は資産収益に影響してないという意味? Price Taker?)

$J$関数は投資家が当期と未来の意思決定の関数ではありません。(???)
これらの意思決定はライフサイクルで期待効用最大化をすべきですが、J関数は各スナップショット/断面での意思決定であります。
(本来であれば、各期間の意思決定で、全体のライフサイクル期待効用を最大化すべきですが、
$J$関数では、各期までの情報で意思決定を修正しているという意味????)

Backwardでこの問題を解く。最後の期間の意思決定を考えると、以前の解いた単期間モデルになる。
つまり、$T-1$までの情報で、第$T$期の消費投資選択を解く。順次に前の期間にも適応して、第$0$期までの
最適解を解くことは可能。
\subsection*{5.2.1 最後の期間}
最後の期間では、$J$は遺産だげになる。
その前の$T-1$期では最後の意思決定をする。
5.6式では期末資産は$T-1$期の消費と投資決定によることを示した。
その中の貯蓄水準$S_{T-1}$と収益$R_{T-1}$は5.1.2で定義したものです。
5.6式は一期間モデルになる。
5.6式を$C_{T-1}$と$w_{i,T-1}^*$で微分して、0に置く。
すると5.7と5.8の$n+1$本の式を得る。

5.9式を得ると、 5.8式と5.9式で投資家の意思を決定する。
これは第4章の4.6式と4.10式と比べれば、 遺産関数Bで期末効用関数Uを代替した。
最適解を5.6式に用いて、$W_{T-1}$に微分すれば、5.10式を得る。

(5.10式の変形はわかりません。)

5.11式のenvelope条件を得る。
当期の消費の限界効用は以後の将来の消費をの限界効用に等しい。
\subsection*{5.2.2 Bellman方程式}
更に、一期前の$T-2$では$J$関数は5.12式ですが、投資家は関心しているのは
$U(C_{T-1},T-1)+B(W_T,T)$
の期待値最大化。その値は$C_{T-1}$と${w_{i,T-1}}$に依存する。
$C_{T-1}$と${w_{i,T-1}}$は最適化原則で決められる。

最初の決定を固定して、 残りの決定は最初の決定には最適でなければならない。

5.12式のmaxは、残りの全期間の決定ですが、最適化原則では
$T-2$期の決定に関わらず、その決定をしたら、残りの$T-1$期の決定は最適でなければならない。

(5.13式からわかりません。)
\subsection*{5.2.3 一般解}
5.15式の最適化条件は5.16式で、
毎期にはBellman方程式を5.18式のように書きます。

第$t$期の最適化条件は5.19式。
5.19式と5.20式のモデルは一期間モデルに似て、
当期の消費の限界効用は以後の将来の消費をの限界効用に等しい。
ポートフォリオの配分はすべての資産の限界効用のウィートの資産収益の期待値は等しくなるべき。

ただし、毎期の消費水準と投資比率を求めるには困難で、最適化条件はJ関数に依存することで、
最適化解は将来の投資・収入・効用にも依頼する。

最後期から順次一期前に解答をとる。

最適解の表現式は必ずしも求められるではないですが、毎期の一回条件の解析解には必要です。
しかし、効用関数・賃金収入・資産収益に適切な仮定を置けば、解析解を求められる。

\end{document}
